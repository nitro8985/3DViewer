\documentclass{article}
\usepackage[utf8]{inputenc}
\title{3DViewer v1.0}
\author{jdoomfis, aremedio, hnathani. (c) 21-school.ru}
\date{June 2022}

\begin{document}
\maketitle

\section{General}
3DViewer v1.0 is a GUI application for viewing 3D wireframe models. The models themselves must be loaded from .obj files and be viewable on the screen with the ability to rotate, scale and translate.
\section{Installation}
You can build the programm right in the current directory from terminal by running \textbf{make viewer} , then run it as \textbf{./viewer}.
Default installation: \textbf{make install}.
To install it to custom directory run \textbf{./install.sh -t 'dirname'}.
To uninstall use \textbf{make uninstall} or \textbf{./unistall.sh -f 'dirname'}.

\section{Usage}
Main window of the application contains :
\begin{itemize}
\item A button to select the model file and a field to output its name.
\item A visualisation area for the wireframe model.
\item Button/buttons and input fields for translating the model.
\item Button/buttons and input fields for rotating the model.
\item Button/buttons and input fields for scaling the model.
\item Information about the uploaded model - file name, number of vertices and edges.
\end{itemize}

\end{document}
